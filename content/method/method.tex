% include the figures path relative to the master file
\graphicspath{ {./content/method/figures/} }

\section{Experimental Setup}\label{sec:exp}

The experimental set-up is summarized in Table~\ref{tab:survey-tab}, where the most relevant works in Sect.\,\ref{sec:review} are formulated as the 5-steps standard classification procedure described in Fig.\,\ref{fig:ML-scheme}.

\subsection{Implementation details}\label{sec:exp:implementation}
The experiments, described in this work, are publicly available at~\cite{rethinopaty20016apr-repoICPR} allowing for further comparisons and improvements.
All the methods in Table~\ref{tab:survey-tab} have been developed using \emph{protoclass}~\cite{protoclass2016apr-repoICPR}, a rapid prototyping toolkit to perform image processing and \gls{ml} tasks.
Furthermore, each method has been implemented as a plug-in to~\cite{rethinopaty20016apr-repoICPR}, so that all methods can be evaluated in a common framework
\footnote{See table\,\ref{tab:survey-tab} for standalone repositories of each method. All repositories provide tests to ensure that our implementation comply with the original work.}.

Note that Liu~\emph{et~al.} train the algorithm at the B-scan level, and \gls{seri} dataset provides \gls{gt} at volume level only.
Thus, two strategies have been explored to solve this issue:
(i) similarly to Srinivasan~\emph{et~al.}, at training stage, all B-scans are considered as abnormal for a DME volume and at testing stage, a majority vote rule is applied to whether label a volume as abnormal or not;
(ii) similarly to Venhuizen~\emph{et~al.}, an approach using BoW is used.
% The strategy to the best classification performance will be reported in Table~\ref{tab:summary_results}.
From the methods reviewed in Sect.\,\ref{sec:review}, we decline to implement Albarrak~\emph{et~al.} and Anantrasirichai~\textit{et~al.}.
The former do not provide sufficient implementation details to replicate their results~\cite{albarrak2013age};
while, the latter use a descriptor based on the layer thickness which require a layer segmentation stage using a generic segmentation algorithm and further user validation~\cite{anantrasirichai2013svm}.


\subsection{Evaluation}\label{sec:evaluation}
All the experiments are evaluated in terms of \gls{se} and \gls{sp} (see
Fig.\,\ref{fig:evaluation}) using the \gls{ltpocv} strategy.\footnote{the same
  evaluation strategy was applied~\cite{Lemaintre2015miccaiOCT}}
\gls{se} evaluates the performance of the classifier with respect to the
positive class, while the \gls{sp} does the same with respect to negative class.
\gls{ltpocv} keeps two volumes (one normal and one \gls{dme}) for testing while
the remaining volumes are used as training. The advantage of using \gls{ltpocv} over \gls{lopocv} is
that \gls{ltpocv} keeps the training data balanced.  
The main drawback of using \gls{ltpocv} (or \gls{lopocv}) is that despite
reporting robust performance estimators, variance of this descriptors cannot be
computed. 
Despite this limitation, \gls{ltpocv} strategy has been adopted here, 
since regular \gls{cv} cannot be applied due to the reduced size of the dataset.

\colorlet{circle edge}{blue!50}
\colorlet{circle area}{blue!20}


% Definition of circles
% \def\myRadius{1.5cm}
% \def\vennSpace{(0,0) rectangle (6cm,4cm)}
% \def\predictedCircle{(2cm,2cm) circle (\myRadius)}
% \def\actualCircle{(4cm,2cm) circle (\myRadius)}
% \def\myLabelRadius{1.3cm}

\def\myRadius{.75cm}
\def\vennSpace{(0,0) rectangle (3cm,2cm)}
\def\predictedCircle{(1cm,1cm) circle (\myRadius)}
\def\actualCircle{(2cm,1cm) circle (\myRadius)}
\def\myLabelRadius{.60cm}

\tikzset{fillbase/.style={fill=circle area, draw=circle edge, thick},
         filled/.style={pattern=crosshatch dots, draw=circle edge, thick},
         outline/.style={draw=circle edge, thick}}

\def\drawPredicted{
    \draw[outline] \predictedCircle node (x){}; % {$\bullet$};
    \node [above left=\myLabelRadius of x, anchor=center, outer sep=0](p){$P$};
    \node  at (p.300) {$+$};
    \node  at (p.120) {$-$};
}

\def\drawActual{
    \draw[outline] \actualCircle node (x){}; % {$\bullet$};
    \node [above right=\myLabelRadius of x, anchor=center, outer sep=0](a){$A$};
    \node  at (a.60) {$-$};
    \node  at (a.240) {$+$};
}

% Define the different metrics: tp, tn, fp, fn
\def\tp{
      \draw[outline] \vennSpace;
      \begin{scope}
        \clip \predictedCircle;
        \fill[filled] \actualCircle;
      \end{scope}
      \drawPredicted
      \drawActual
      % \draw[outline] (current bounding box.south west)
      %   rectangle (current bounding box.north east);
}

\def\tn{
      \draw[outline] \vennSpace;
  \begin{scope}[even odd rule]
    \fill[filled] \vennSpace
      \actualCircle
      \predictedCircle;
    \clip \actualCircle;
    \fill[white] \predictedCircle;
  \end{scope}
  \drawPredicted
  \drawActual
}

\def\fp{
      \draw[outline] \vennSpace;
      \begin{scope}
        \clip \predictedCircle;
        \fill[filled, even odd rule]
              \predictedCircle \actualCircle;
      \end{scope}
      \draw[outline] \vennSpace;
      \drawPredicted
      \drawActual
}

\def\fn{
      \draw[outline] \vennSpace;
      \begin{scope}
        \clip \actualCircle;
        \fill[filled, even odd rule]
              \actualCircle \predictedCircle;
      \end{scope}
      \draw[outline] \vennSpace;
      \drawPredicted
      \drawActual
}

\def\se{
  \fill[fillbase] \actualCircle;
      \begin{scope}
        \clip \predictedCircle;
        \fill[filled] \actualCircle;
      \end{scope}
      \draw[outline] \vennSpace;
      \drawPredicted
      \drawActual
}


\def\sp{
  \fill[fillbase, even odd rule]
    \vennSpace \actualCircle;
  \begin{scope}[even odd rule]
    \fill[filled] \vennSpace
      \actualCircle
      \predictedCircle;
    \clip \actualCircle;
    \fill[white] \predictedCircle;
  \end{scope}
  \draw[outline] \vennSpace;
  \drawPredicted
  \drawActual
  }


%%% Local Variables: 
%%% mode: latex
%%% TeX-master: "../../../main"
%%% End: 

\begin{figure}

  \def\myRadius{.65cm}
  \def\vennSpace{(0,0) rectangle (2.6cm,1.6cm)}
  \def\predictedCircle{(.8cm,.8cm) circle (\myRadius)}
  \def\actualCircle{(1.8cm,.8cm) circle (\myRadius)}
  \def\myLabelRadius{.450cm}

  \subfigure[][Confusion matrix with truly and falsely positive samples detected (TP, FP) in the first row, from left to right and the falsely and truly negative samples detected (FN, TN) in the second row, from left to right.]{
    \label{fig:evaluation:confusion_matrix}
    \begin{tikzpicture}[scale=0.5]
      \node at (0,0){
        \begin{tabular}{
            >{\centering}m{1em} >{\centering}m{1em} >{\centering}m{1in} >{\centering\arraybackslash}m{1in}}
          % c>{\centering}m{2em}ccc}
          & & \multicolumn{2}{c}{ Actual Class }\\
          & & A+ & A- \\
          % \parbox[t]{2mm}{\multirow{2}{*}{\rotatebox[origin=c]{90}{\usebox \centering Predicted Class}}}& P+ &  \tikz{\tp} & \tikz{\fp} \\
          \multirow{3}{*}{\rotatebox[origin=c]{90}{Predicted Class}}& P+ &  \tikz{\tp} & \tikz{\fp} \\
          & P- & \tikz{\fn} & \tikz{\tn}
        \end{tabular}
      };
    \end{tikzpicture}
  }\\
  \centering
  \subfigure[][\gls{se} and \gls{sp} evaluation, corresponding to the ratio of the doted area over the blue area.]{
    \label{fig:evaluation:roc_axis}
    \begin{tikzpicture}[scale=0.5]
      \def\seEquation{$SE = \frac{TP}{TP+FN}$}
      \def\spEquation{$SP = \frac{TN}{TN+FP}$}
      \node[label={[]below:\seEquation}](se){\tikz{\se}};
      \node[right=5pt of se, label={[]below:\spEquation}]{\tikz{\sp}};
      % \node[label={[]right:\seEquation}](se){\tikz{\se}};
      % \node[below=5pt of se, label={[]right:\spEquation}]{\tikz{\sp}};
    \end{tikzpicture}
  }

  \caption{Evaluation metrics:
    \protect\subref{fig:evaluation:confusion_matrix} confusion matrix,
    \protect\subref{fig:evaluation:roc_axis} \gls{se} - \gls{sp}
  }
  \label{fig:evaluation}
\end{figure}



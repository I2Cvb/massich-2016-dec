% include the figures path relative to the master file
\graphicspath{ {./content/method/figures/} }

\section{Data}\label{sec:data}
Proper comparison of different methodologies require a common dataset to test
these methodologies.
The lack of public data for comparing methodologies is recurrent claim in the
medical image community~\cite{giger2008anniversary}.
To amend this limitation, Duke Univeristy made two \gls{sdoct} datasets
available to the ophtalmic community~\cite{farsiu2014quantitative,Srinivasan2014}.

The rest of this section highlights the advantages and disadvantages of the
datasets provided by Duke University, points out why this datasets cannot be
used for our purposes and finally describes our data: \emph{the \gls{seri} dataset}~\cite{seri2016apr-repoICPR}.

The former dataset from Duke University consists of $384$ \gls{oct} annotated volumes classified
either as \gls{amd} or normal cases.
Despite the advantage of testing in large datasets, this dataset cannot be used to conduct our study since we are interested in \gls{dme} and not \gls{amd}.
This dataset has been used by Venhuizen~\emph{et~al.}~\cite{Venhuizen2015} to
test their method since their main interest is \gls{amd} detection.
The later dataset from Duke Univeristy consists of $45$ pre-processed \gls{oct}
volumes and labeled as \gls{amd}, \gls{dme}, and normal. Despite this dataset is
suitable to our goal of classifying \gls{dme} vs normal volumes, the dataset has
been dropped since there is no access to the original data. All volumes have
been denoised, aligned and cropped. This dataset has been used to conduct the
experimentation reported by Srinivasan~\emph{et~al.}~\cite{Srinivasan2014}.


The dataset to conduct our study has been acquired by the \gls{seri}, using CIRRUS TM (Carl Zeiss Meditec, Inc., Dublin, CA) \gls{sdoct} device.
This dataset consists of 32 \gls{oct} volumes, subdivided into 16 \gls{dme} and 16 normal cases.
Each volume contains $128$ B-scans with a resolution of \SI[product-units=repeat]{512x1024}{\px}.
All \gls{sdoct} images have been read and assessed by trained graders and identified as normal or \gls{dme} cases, based on evaluation of retinal thickening, hard exudates, intraretinal cystoid space formation and subretinal fluid (see Fig.\,\ref{fig:bbdd}).

% 4, 2, 1\\
% 1, 1, 1\\
% 3, 2, 1\\

\section{Results and Discussion}\label{sec:results}\label{sec:discussion}

% Presenting the results
The entire set of experimental results can be found in~\cite{rethinopaty20016apr-repoICPR}, while Table~\ref{tab:summary_results} shows the configuration leading to the best results of each method.
The results are reported in terms of \gls{se} and \gls{sp} (see Sect.\,\ref{sec:evaluation}), and sorted left to right in descending order.

% Presenting which configuration gives best results
Lemaitre~\emph{et~al.} achieves the best results when using \gls{lbptop} features, a global mapping and histogram representation~\cite{Lemaintre2015miccaiOCT}.
Sankar~\emph{et~al.} performs better when using \gls{hog} features, and \gls{pca} representation~\cite{Alsaih2016apr-repoICPR}.
From the modifications of Liu~\emph{et~al.}, proposed in Sect\,\ref{sec:experiment}, the best results are achieved when using majority voting.
See table~\ref{tab:survey-tab} for rest of configuration details.

% Comparison Details
Results in~\cite{rethinopaty20016apr-repoICPR} indicate two decisive factors in development of the methodologies:
(i) features describing the entire volume rather than each B-scan are more discriminative.
(ii) a pre-processing stage with denoising is fundamental.

% More comparison details but less important
Other observations include the facts that
(i) when representing B-scans, local mapping in conjunction with dimensionality reduction, either using \gls{pca} or \gls{bow}, improve the results.
However, the combination of both decreases the performance in comparison to non reduced histogram representation.
(ii) building \gls{bow} models from concatenated detected features, might lead to the curse of dimensionality, which would explain why the \gls{rbf}-\gls{svm} overfits in~\cite{Alsaih2016apr-repoICPR, liu20016apr-repoICPR}.


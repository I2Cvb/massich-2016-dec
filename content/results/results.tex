\section{Results and Discussion}\label{sec:results}\label{sec:discussion}

% Presenting the results
The entire set of experiments with their associated results can be found in~\cite{rethinopaty20016apr-repoICPR}, while Table~\ref{tab:survey-tab} shows the configuration leading to the best results of each method.
The results are reported in terms of \gls{se} and \gls{sp} (see Sect.\,\ref{sec:evaluation}).

% Presenting which configuration gives best results
Lemaitre~\emph{et~al.} achieve the best results when using \gls{lbptop} features, a global mapping, and histogram representation~\cite{Lemaintre2015miccaiOCT}.
Alsaih~\emph{et~al.} perform better when using \gls{hog} features with \gls{pca} representation~\cite{Alsaih2016apr-repoICPR}.
Our interpretation of Liu~\emph{et~al.}, as proposed in Sect.\,\ref{sec:exp}, achieves the best results when using majority voting instead of \gls{bow} models.
Refer to Table~\ref{tab:survey-tab} for configuration details of the remaining methods.

% Comparison Details
Results in~\cite{rethinopaty20016apr-repoICPR} indicate two main findings with major impact:
(i) features describing the entire volume rather than each B-scan are more discriminative; and
(ii) a pre-processing stage with denoising is fundamental.

% More comparison details but less important
Other observations include the facts that
(i) to represent B-scans, local mapping in conjunction with dimension reduction, either using \gls{pca} or \gls{bow}, improve the results.
However, the combination of both decreases the performance in comparison to non reduced histogram representation;
(ii) building \gls{bow} models from concatenated detected features, might lead to the curse of dimensionality, which would explain why the \gls{rbf}-\gls{svm} overfits in~\cite{liu20016apr-repoICPR};
(iii) building \gls{bow} models from the concatenation of all features for each B-scan, might lead to the curse of dimensionality since $128$ samples per volume is not enough to describe a space with a number of dimensions of the order of thousands; which could explain the over-fitting using \gls{rbf}-\gls{svm} as in~\cite{liu20016apr-repoICPR}.

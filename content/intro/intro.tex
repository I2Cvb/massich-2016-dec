% include the figures path relative to the master file
\graphicspath{ {./content/intro/figures/} }

\section{Introduction}
\label{sec:intro}  % \label{} allows reference to this section

\gls{dr}, and more particularly \gls{dme}, are leading causes of irreversible vision loss and the most common eye diseases in individuals with diabetes.
Taking into account that the number of individuals affected by diabetes dieases are expected to grow exponentially in the next decade~\cite{wild2004global},
developing methodologies for early detection and treatment of \gls{dr} and \gls{dme} has become a priority to prevent adverse effects.

The main focus of this work is to describe the actual state of \gls{dme} detection in \gls{oct} images.
\gls{dme} presents an increase in retinal thickness within 1 disk diameter of the fovea center with or without hard exudates and sometimes associated with cysts~\cite{ETDRSG1985}.
\gls{sdoct} is an emerging eye imaging modality providing cross-sectional retinal morphology information~\cite{Wang2015}, which cannot be estimated from more established eye imaging modalities such as fundus imaging.

The initial efforts of the ophtalmic community in developing technologies for \gls{sdoct} have been placed in segmenting the retinal layers, which is a necessary step for retinal thickness measurements~\cite{Chiu2010,Kafieh2013}.
However, latter efforts address the specific problem of \gls{dme} automatic detection in \gls{oct} volumes.
These efforts reveal the needs to address: (i) enhancing the quality of \gls{oct} volumes,
(ii) finding pathology signs,
and (iii) appropriate classification strategies.

Advances in any of those regards is of great interest since (i) manual evaluation of \gls{sdoct} volumetric scans is expensive and time consuming~\cite{Venhuizen2015};
(ii) \gls{sdoct} acquisition has some shortcomings due to eye movements during the scanning~\cite{Liu2011}, reflectivity nature of the retina~\cite{schuman2004optical}, high level of noise and inconsistent quality of the images;
(iii) due to the coexistence of multiple pathologies~\cite{Liu2011} as well as large intra-pathology variability, consistently identifying pathology-specific biomarkers remains challenging~\cite{Venhuizen2015}.

The rest of this article is structured as follows: Section~\ref{sec:review} offers a general idea of the literature state-of-the-art in \gls{sdoct} volume classification. Section~\ref{sec:data} reviews some publicly available datasets and states the need for another one that suits the classification task here described.
Section~\ref{sec:exp} proposes an experimental benchmark to compare different methodologies presented in Sect.\,\ref{sec:review}.
Section~\ref{sec:results} reports and discusses the obtained results, while Sect.\,\ref{sec:conclusion} wraps up our thoughts regarding this work and its possible direction.

% Some stuff that emac's colegues use
%%% Local Variables:
%%% mode: late
%%% TeX-master: "../../main.tex"
%%% End: \section{introduction}


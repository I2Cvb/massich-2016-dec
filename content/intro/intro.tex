% include the figures path relative to the master file
\graphicspath{ {./content/intro/figures/} }

\section{Introduction}
\label{sec:intro}  % \label{} allows reference to this section

\gls{dr} and \gls{dme}, as common eye diseases, are leading causes of irreversible vision loss in individuals with diabetes.
The number of individuals affected by diabetes dieases are expected to grow exponentially in the next decade. 
Therefore the need for early detection and treatment of \gls{dr} and \gls{dme} is becoming a major health issue.
%Number of people affected by diabetes are expected to grow exponentionally affecting over \SI{300}{M} pople worldwide by 2025~\cite{Wild2004}.
%Just in United States alone, health care and associated costs related to eye diseases are estimated at almost \SI{500}[\$]{M}~\cite{Sharma2005}.
%Moreover, the prevalent cases of \gls{dr} are expected to grow exponentially affecting over \SI{300}{M} people worldwide by 2025~\cite{Wild2004}.
%Given this scenario, early detection and treatment of \gls{dr} and \gls{dme} play a major role to prevent adverse effects such as blindness.

%\gls{dme} is characterized as an increase in retinal thickness within 1 disk diameter of the fovea center with or without hard exudates and sometimes associated with cysts~\cite{ETDRSG1985}.
\gls{dme}, as the main focus of this article, is characterized as an increase in retinal thickness within 1 disk diameter of the fovea center with or without hard exudates and sometimes associated with cysts~\cite{ETDRSG1985}.
For many years, fundus images have been the modality of choice to reveal most of eye pathologies~\cite{Mookiah20132136,Trucco2013}.
However, \gls{oct} has recently shown to provide useful additional information about cross-sectional retinal morphology~\cite{Wang2015}, reflected by the growing interest in developing methodologies for this modality.
In this sense, great efforts have been placed in retinal layers segmentation, which is a necessary step for retinal thickness measurements~\cite{Chiu2010,Kafieh2013}.
However, few studies have addressed the specific problem of \gls{dme} automatic detection in \gls{oct} volumes, leaving a large ground to be covered in terms of: (i) manipulating \gls{oct} volumes, (ii) finding pathology signs, or (iii) appropriated classification strategies.

Advances in any of those regards is of great interest since (i) manual evaluation of \gls{sdoct} volumetric scans is expensive and time consuming~\cite{Venhuizen2015}.
%(ii) \gls{sdoct} acquisition has some deficiencies due to eye movement during the scan~\cite{Liu2011}, the reflectivity nature of the retina~\cite{schuman2004optical}, the fact that \gls{oct} suffers from high levels of noise and the overall image quality is inconsistent~\cite{barnum2008local}.
(ii) \gls{sdoct} acquisition has some shortcomings due to eye movement in the scan~\cite{Liu2011}, reflectivity nature of the retina~\cite{schuman2004optical}, high level of noise and inconsistent quality of the images. 
%(iii) Coexistence of multiple pathologies~\cite{Liu2011}, easy to miss pathology signs~\cite{Venhuizen2015}, or large variability within the pathology treats which difficult to obtain proper radiomix to facilitate the task of classification.
(iii) Due to coexistence of multiple pathologies~\cite{Liu2011} and large variability within the pathology treates it is difficult to obtain the proper sign for each pathology and moreover it is easy to miss them~\cite{Venhuizen2015}.


The rest of this article is structured as follows: Section~\ref{sec:review} offers a general idea of the literature state-of-the-art in \gls{sdoct} volume classification. Section~\ref{sec:data} reviews some publicly available datasets and states the need for another one that suits the classification task here described.
Section~\ref{sec:exp} proposes an experimental benchmark to compare different methodologies presented in Sect.\,\ref{sec:review}.
Section~\ref{sec:results} reports and discusses the obtained results, while Sect.\,\ref{sec:conclusion} wraps up our thoughts regarding this work and its possible direction.

% Some stuff that emac's colegues use
%%% Local Variables:
%%% mode: late
%%% TeX-master: "../../main.tex"
%%% End: \section{introduction}


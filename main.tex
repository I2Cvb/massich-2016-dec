% Template for IEEEtransactions
% Adapted by Sik in March 2016 to meet his requirements
%
\documentclass[conference,twocolumn,letterpaper,10pt]{latex/IEEEtran}
%% Latex documents that need direct input
%
% In order to include files without having a clear page using \include*,
% the newclude package is required
\usepackage{newclude}

% Include acroyms
\usepackage[acronym]{glossaries}

% Use biblatex to manage the referencing
%
% \usepackage[style=reading,backend=biber]{biblatex}
% \usepackage[authoryear, round, sort]{natbib}
  % contains the latex packages for IEEEtrans
%  The following command loads a graphics package to include images
%  in the document. It may be necessary to specify a DVI driver option,
%  e.g., [dvips], but that may be inappropriate for some LaTeX
%  installations.
\usepackage[]{graphicx}

% Use nameref to cite supporting information files (see Supporting Information section for more info)
\usepackage{nameref,hyperref}

% line numbers
\usepackage[right]{lineno}

% Use biblatex to manage the referencing
%
% \usepackage[style=reading,backend=biber]{biblatex}
% \usepackage[authoryear, round, sort]{natbib}

% Clever cross referencing. Using cleverref, instead of writting
% figure~\ref{...} or equation~\ref{...}, only \cref{...} is required.
% The package interprates the references and introduces the figure, fig.,
% equation, eq., etc keywords. \Cref forces first letter capital.
% >> WARNING: This package needs to be loaded after hyperref, math packages,
%             etc. if used.
%             Cleveref is recomended to load late
\usepackage{cleveref}

% To create random text use lipsum
\usepackage{lipsum}
        % contains the latex packages
% To create random text use lipsum
\usepackage{lipsum}

% See the number of line
\usepackage[switch,columnwise]{lineno}
\modulolinenumbers[5]

% Managing TODOES and unfinished figures
\usepackage{todonotes}

% Some packages useful for edition
\usepackage{changebar}
\usepackage{changes}

% Change the top rule
\newcommand*\ctoprule[1]{\addlinespace\cmidrule[\heavyrulewidth]{#1}}
% contains the latex packages
% \usepackage[numbers]{natbib}
%%%%%%%%%%%%%%%%%%%%%%%%%%%%%%%%%%%%%%%%%%%%%%%%%%%%%%%%%%%%% 
% Custom setup
%%%%%%%%%%%%%%%%%%%%%%%%%%%%%%%%%%%%%%%%%%%%%%%%%%%%%%%%%%%%% 
%>>>> uncomment following for page numbers
% \pagestyle{plain}    
%>>>> uncomment following to start page numbering at 301 
%\setcounter{page}{301} 

% Generates the acronyms list
\makeglossaries

        % contains package and variables init.
%!TEX root = ../main.tex

%% Acronym definition example using glossaries package
%% \usepackage{acro} cannot be used with IEEEtrans use Tobias Oetiker’s {acronym}
%% The acronym environment will have a problem with IEEEtran because of the
%% modified IEEE style description list environment. The optional argument of
%% the acronym environment cannot be used to set the width of the longest label.
%% A workaround is to use \IEEEiedlistdecl to accomplish the same thing:
%%
%% \renewcommand{\IEEEiedlistdecl}{\IEEEsetlabelwidth{S
%% ONET}}
%% \begin{acronym}
%% .
%% .
%% \end{acronym}
%% \renewcommand{\IEEEiedlistdecl}{\relax}% reset back

\newacronym{us}{US}{Ultra-Sound}
\newacronym{cad}{CAD}{Computer Aided Diagnosis}
\newacronym{dm}{DM}{Digital Mammography}
\newacronym{gt}{GT}{Ground Truth}
\newacronym{ml}{ML}{Machine Learning}
\newacronym{acm}{ACM}{Active Contour Model}
\newacronym{crf}{CRFs}{Conditional Random Fields}
\newacronym{mrf}{MRFs}{Markov Random Fields}
\newacronym{cv}{CV}{Computer Vision}
\newacronym{icm}{ICM}{Iterated Conditional Modes}
\newacronym{sa}{SA}{Simulate Anealing}
\newacronym{gc}{GC}{Graph-Cuts}
\newacronym{birads}{BI-RADS}{Breast Imaging-Reporting and Data System}
\newacronym{mad}{MAD}{Median Absolute Deviation}
\newacronym{qc}{QC}{Quadratic-Chi}
\newacronym{sift}{SIFT}{Self-Invariant Feature Transform}
\newacronym{bof}{BoF}{Back-of-Features}
\newacronym{acr}{ACR}{American College of Radiology}
\newacronym{fa}{FA}{Fibro-Adenoma}
\newacronym{dic}{DIC}{Ductal Inflating Carcinoma}
\newacronym{ilc}{ILC}{Inflating Lobular Carcinoma}
\newacronym{fpr}{FPR}{False Positive Ratio}
\newacronym{fnr}{FNR}{False Negative Ratio}
\newacronym{fn}{FN}{False Negative}
\newacronym{fp}{FP}{False Positive}
\newacronym{rbf}{RBF}{Radial Basis Function}
\newacronym{dr}{DR}{Diabetic Retinopathy}
\newacronym{dme}{DME}{Diabetic Macular Edema}
\newacronym{oct}{OCT}{Optical Coherence Tomography}
\newacronym{sdoct}{SD-OCT}{Spectral Domain OCT}
\newacronym{amd}{AMD}{Age-related Macular Degeneration}
\newacronym{hog}{HOG}{Histogram of Oriented Gradients}
\newacronym{svm}{SVM}{Support Vector Machines}
\newacronym{bow}{BoW}{Bag-of-Words}
\newacronym{rf}{RF}{Random Forest}
\newacronym{tp}{TP}{True Positive}
\newacronym{tn}{TN}{True Negative}
\newacronym{roc}{ROC}{Receiver Operating Characteristic}
\newacronym{auc}{AUC}{Area Under the Curve}
\newacronym{lbp}{LBP}{Local Binary Patterns}
\newacronym{pca}{PCA}{Principal Component Analysis}
\newacronym{nlm}{NLM}{Non-Local Means}
\newacronym{lopocv}{LOPO-CV}{Leave-One-Patient Out Cross-Validation}
\newacronym{lbptop}{LBP-TOP}{LBP from Three Orthogonal Planes}
\newacronym{se}{SE}{Sensitivity}
\newacronym{sp}{SP}{Specificity}
\newacronym{sw}{P}{patch}
\newacronym{nn}{NN}{Nearest Neighbor}
\newacronym{gb}{GB}{Gradient Boosting}
\newacronym{lr}{LR}{Logistic Regression}
\newacronym{adb}{AdB}{AdaBoost}
\newacronym{acc}{ACC}{Accuracy}
\newacronym{f1}{F1}{F1-score}
\newacronym{nf}{NF}{non-flatten}
\newacronym{f}{F}{flatten}
\newacronym{fal}{F+A}{flatten-aligned}
\newacronym{fac}{F+A+C}{flatten-aligned-cropped}
\newacronym{rpe}{RPE}{Retinal Pigment Epithelium}
\newacronym{gmm}{GMM}{Gaussian Mixture Model}
\newacronym{voi}{VOI}{volume of interest}
\newacronym{glcm}{GLCM}{Gray-level co-occurrence matrix}
\newacronym{seri}{SERI}{Singapore Eye Research Institute}
\newacronym{bm3d}{BM3D}{Block Matching 3D filtering}
\newacronym{ltpocv}{LTPO-CV}{Leave-Two-Patient Out Cross-Validation}      % contains the acronims

%% Select inputing only one part of the document
%\includeonly{content/intro/intro}   % the file wihtout .tex
%\includeonly{content/other/other_content}

% \addbibresource{./content/lit_review.bib}
% \addbibresource{./content/biblatex-examples.bib}

%% Include all macros below

\newcommand{\lorem}{{\bf LOREM}}
\newcommand{\ipsum}{{\bf IPSUM}}

%% END MACROS SECTION

\begin{document}
% Article Title
\def \ArticleTitle{Article Title}

% Author(s) Name(s)
\def \AuthorA{Joan~Massich}
\def \AuthorB{Mojdeh~Rastgoo}
\def \AuthorC{Guillaume~Lema\^itre}
\def \AuthorD{Carol Y. Cheung}
\def \AuthorE{Tien Y. Wong}
\def \AuthorF{D\'esir\'e~Sidib\'e}
\def \AuthorG{Fabrice~M\'eriaudeau}

% Author(s) Email(s)
\def \AuthorAemail{joan.massich@u-bourgogne.fr}
\def \AuthorBemail{mojdeh.rastgoo@gmail.com}
\def \AuthorCemail{g.lemaitre58@gmail.com}

% Institution(s) Name(s)
\def \InstitutionA{LE2I UMR6306, CNRS, Arts et M\'etiers, Univ. Bourgogne Franche-Comt\'e,\\ 12 rue de la Fonderie, 71200 Le Creusot, France}
\def \InstitutionB{ViCOROB, Universitat de Girona, Campus Montilivi, Edifici P4, 17071 Girona, Spain}
\def \InstitutionC{Singapore Eye Research Institute, Singapore National Eye Center, Singapore}
\def \InstitutionD{Centre for Intelligent Signal and Imaging Research (CISIR), Electrical \& Electronic Engineering Department,\\ Universiti Teknologi Petronas, 32610 Seri Iskandar, Perak, Malaysia}

% Article title
\title{Classifying DME vs Normal SD-OCT volumes: A review}

\author{
\IEEEauthorblockN{\AuthorA\IEEEauthorrefmark{1}, \AuthorB\IEEEauthorrefmark{1}\IEEEauthorrefmark{2}, \AuthorC\IEEEauthorrefmark{1}\IEEEauthorrefmark{2}, \AuthorD\IEEEauthorrefmark{1}, \\ \AuthorE\IEEEauthorrefmark{2}, \AuthorF\IEEEauthorrefmark{2}, \AuthorG\IEEEauthorrefmark{1}\IEEEauthorrefmark{4}}
\IEEEauthorblockA{\IEEEauthorrefmark{1}\InstitutionA}
\IEEEauthorblockA{\IEEEauthorrefmark{2}\InstitutionB}
\IEEEauthorblockA{\IEEEauthorrefmark{3}\InstitutionC}
\IEEEauthorblockA{\IEEEauthorrefmark{4}\InstitutionD}
}             % contains the Title and Autor info

% Variables file
%!TEX root = ./article.tex

% Article Title
\def \ArticleTitle{Article Title}

% Author(s) Name(s)
\def \AuthorA{Author Name}

% Author(s) Email(s)
\def \AuthorAemail{EmailA}

% Institution(s) Name(s)
\def \InstitutionA{Institution Name}

% Article Title
\newcommand {\Title} {\ArticleTitle}

% Authors
\newcommand {\Authors} {\IEEEauthorblockN{\AuthorA}}

% Institution
\newcommand {\Institutions} {\IEEEauthorblockA{\InstitutionA}\\
                            Email: \AuthorAemail}

% Variable to control if the bibliography must be include
\def \hasBibliography{0}


\maketitle

% Please keep the abstract below 300 words
\begin{abstract}
  \added[id=old]{
    This paper addresses the problem of automatic classification of \gls{sdoct}
    data for automatic identification of patients with \gls{dme} versus normal
    subjects.  \gls{oct} has been a valuable diagnostic tool for \gls{dme},
    which is among the most common causes of irreversible vision loss in
    individuals with diabetes.  Here, a classification framework with five
    distinctive steps is proposed and we present an extensive study of each
    step.  Our method considers combination of various pre-processings in
    conjunction with \gls{lbp} features and different mapping strategies.
    Using linear and non-linear classifiers, we tested the developed framework
    on a balanced cohort of 32 patients.
  }

  \added[id=old]{
    Experimental results show that the proposed method outperforms the previous
    studies by achieving a \gls{se} and \gls{sp} of 81.2\% and 93.7\%,
    respectively.  Our study concludes that the 3D features and high-level
    representation of 2D features using patches achieve the best results.  However,
    the effects of pre-processing is inconsistent with respect to different
    classifiers and feature configurations.
  }
\end{abstract}

\begin{IEEEkeywords}
\end{IEEEkeywords}

\linenumbers

%% Incldue the content without .tex extension
% \acresetall  % reset the acronyms from the abstract
\include*{content/intro/intro}          % the file wihtout .tex
\include*{content/survey/background}
\include*{content/method/method}
\include*{content/results/results}
\section{Discussion}\label{sec:discussion}
\section{Conclusion and Further work}\label{sec:conclusion}
\nolinenumbers

%\section*{References}
% Either type in your references using
% \begin{thebibliography}{}
% \bibitem{}
% Text
% \end{thebibliography}
%
% OR
%
% Compile your BiBTeX database using our plos2015.bst
% style file and paste the contents of your .bbl file
% here.

% % Imports the bibliography file "sample.bib"
% \bibliography{sample}

\if\hasBibliography 1
% Bibliography
\bibliographystyle{IEEEtran}
% Bibliography file
\bibliography{IEEEabrv,content/bib/literature_review,content/bib/medical/medical,content/bib/retinopathy/retinopathy}
\fi



\end{document}
